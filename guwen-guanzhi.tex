\documentclass[openany,a5paper]{utbook}

\usepackage{ctexheading}
\usepackage{zhnumber}
\ctexset{
  chapter = {
    name        = 古文觀止卷之 ,
    number      = \zhnum{chapter} ,
    format      = \huge\bfseries ,
    nameformat  = {} ,
    titleformat = {} ,
    aftername   = \quad ,
    beforeskip  = 0 pt ,
    afterskip   = 20 pt ,
    afterindent = true ,
    pagestyle   = main ,
  } ,
  section = {
    numbering   = false ,
    indent      = 42 pt ,
    afterindent = true ,
  } ,
}
\zhnumsetup{style=Traditional}
\usepackage{titleps}
\makeatletter
    \newpagestyle{main}{%
      \widenhead{30pt}{30pt}%
      % TODO -- Set the headings of even pages
      \sethead[][][%
      {\if@mainmatter
        \raisebox{\dimexpr-\height-\headsep\relax}[0pt]{%
          \hbox to \textheight{%
            \tate\hspace{3zh}\CTEXthechapter\hfill
            \zhdigits{\arabic{page}}\hspace{4zh}}}%
        \fi}]{%
        \if@mainmatter
        \raisebox{\dimexpr-\height-\headsep\relax}[0pt]{%
          \hbox to \textheight{%
            \tate\hspace{3zh}\CTEXthechapter\hfill
            \zhdigits{\arabic{page}}\hspace{4zh}}}%
        \fi}{}{}
    }\makeatother
\makeatletter

\pagestyle{main}

% adapted from Shinsaku Fujita http://xymtex.com/fujitas2/texlatex/
\makeatletter
\def\hidirikasen#1{%
\ifydir\overline{#1}%
\else\if@rotsw\overline{#1}\else
\setbox\z@\hbox{#1}\leavevmode\lower.7zw
\hbox to\z@{\vrule\@width\wd\z@ \@depth\z@ \@height.4\p@\hss}%
\box\z@
\fi\fi}
\makeatother

% \usepackage{textcomp}
\usepackage{accents}
\makeatletter
\long\def\@fordbleq#1:=#2\do#3{%
  \expandafter\def\expandafter\@fortmp\expandafter{#2}%
    \ifx\@fortmp\@empty \else%
    \expandafter\@dbleql@@p#2==\@nil==\@nil\@@#1{#3}\fi}
\long\def\@dbleql@@p#1==#2==#3\@@#4#5{\def#4{#1}\ifx #4\@nnil \else%
       #5\def#4{#2}\ifx #4\@nnil \else#5\@idbleql@@p #3\@@#4{#5}\fi\fi}
\long\def\@idbleql@@p#1==#2\@@#3#4{\def#3{#1}\ifx #3\@nnil%
%       \let\@nextwhile=\@fornoop \else%
       \expandafter\@fornoop \else%
%      #4\relax\let\@nextwhile=\@idbleql@@p\fi\@nextwhile#2\@@#3{#4}}
      #4\relax\expandafter\@idbleql@@p\fi#2\@@#3{#4}}

\def\namisen{\lower.4zw\hbox{$\underaccent{\tilde}{\null\kern.26zw}$}}
\def\hidirinamisen#1{%
\@tempcnta=\z@\relax
\@fordbleq\member:=#1\do{\global\advance\@tempcnta\@ne}%
\ifnum\@tempcnta=1\relax
\setbox\z@=\hbox{#1}\leavevmode
\hbox to\z@{\hskip0.2zw\hbox to\wd\z@{%
\leaders\hbox to 0.26zw{\hfil\namisen\hfil}\hfill}\hss}%
\box\z@
\else
 \@tempcntb=0
 \@fordbleq\member:=#1\do{%
 \advance\@tempcntb\@ne
 \ifnum\@tempcntb=1\relax
   \setbox\z@=\hbox{\member}\leavevmode
   \hbox to\z@{\hskip0.2zw\hbox to\wd\z@{%
   \leaders\hbox to 0.26zw{\hfil\namisen\hfil}\hfill}\hss}%
   \box\z@\break
  \else
   \ifnum\@tempcntb=\@tempcnta
    \setbox\z@=\hbox{\member}\leavevmode
    \hbox to\z@{\hskip0.2zw\hbox to\wd\z@{%
    \leaders\hbox to 0.26zw{\hfil\namisen\hfil}\hfill}\hss}%
    \box\z@
   \else
    \setbox\z@=\hbox{\member}\leavevmode
    \hbox to\z@{\hskip0.2zw\hbox to\wd\z@{%
    \leaders\hbox to 0.26zw{\hfil\namisen\hfil}\hfill}\hss}%
    \box\z@\break
   \fi\fi}%
 \fi}%
\def\hnijusen{%
\lower.65zw\hbox to.1zw{\hss
\vrule\@width.1zw \@depth\z@ \@height.4\p@\hss}%
\kern-.1zw
\lower.8zw\hbox to.1zw{\hss
\vrule\@width.1zw \@depth\z@ \@height.4\p@\hss}}
%
\def\hidirinijusen#1{%
\@tempcnta=\z@\relax
\@fordbleq\member:=#1\do{\global\advance\@tempcnta\@ne}%
\ifnum\@tempcnta=1\relax
\setbox\z@=\hbox{#1}\leavevmode
\hbox to\z@{\kern.1zw\hbox to\wd0{%
\leaders\hbox{\hfil\hnijusen\hfil}\hfill}\hss}%
\box\z@
\else
 \@tempcntb=0\relax
 \@fordbleq\member:=#1\do{%
 \advance\@tempcntb\@ne
 \ifnum\@tempcntb=1\relax
   \setbox\z@=\hbox{\member}\leavevmode
   \hbox to\z@{\kern.1zw\hbox to\wd0{%
   \leaders\hbox{\hfil\hnijusen\hfil}\hfill}\hss}%
   \box\z@\break
  \else
   \ifnum\@tempcntb=\@tempcnta
   \setbox\z@=\hbox{\member}\leavevmode
   \hbox to\z@{\kern.1zw\hbox to\wd0{%
   \leaders\hbox{\hfil\hnijusen\hfil}\hfill}\hss}%
   \box\z@
   \else
   \setbox\z@=\hbox{\member}\leavevmode
   \hbox to\z@{\kern.1zw\hbox to\wd0{%
   \leaders\hbox{\hfil\hnijusen\hfil}\hfill}\hss}%
   \box\z@\break
   \fi\fi}%
 \fi}%
\makeatother

\usepackage[dvipdfmx,bookmarksnumbered]{hyperref}
\AtBeginShipoutFirst{\special{pdf:tounicode UTF8-UTF16}}

\DeclareKanjiFamily{JY2}{song}{}
\DeclareFontShape{JY2}{song}{m}{n}{<->upschrm-h}{}
\DeclareFontShape{JY2}{song}{bx}{n}{<->ssub*hei/m/n}{}
\DeclareKanjiFamily{JY2}{hei}{}
\DeclareFontShape{JY2}{hei}{m}{n}{<->upschgt-h}{}
\DeclareFontShape{JY2}{hei}{bx}{n}{<->ssub*hei/m/n}{}
\DeclareKanjiFamily{JT2}{song}{}
\DeclareFontShape{JT2}{song}{m}{n}{<->upschrm-v}{}
\DeclareFontShape{JT2}{song}{bx}{n}{<->ssub*hei/m/n}{}
\DeclareKanjiFamily{JT2}{hei}{}
\DeclareFontShape{JT2}{hei}{m}{n}{<->upschgt-v}{}
\DeclareFontShape{JT2}{hei}{bx}{n}{<->ssub*hei/m/n}{}
\AtBeginShipoutFirst{%
  \special{pdf:mapline upstsl-h UniGB-UTF16-H simsun.ttc}%
  \special{pdf:mapline upstsl-v UniGB-UTF16-V simsun.ttc}%
  \special{pdf:mapline upstht-h UniGB-UTF16-H simhei.ttf}%
  \special{pdf:mapline upstht-v UniGB-UTF16-V simhei.ttf}}

\renewcommand\mcdefault{song}
\renewcommand\gtdefault{hei}

\renewcommand{\thefootnote}{〔\zhdigits{\arabic{footnote}}〕}
\setlength\parindent{2zw}

\usepackage{pxrubrica} 
\begin{document}
\mainmatter

\chapter{}

\section{{\hidirikasen{鄭伯}}克\hidirikasen{段}于\hidirinijusen{鄢}}

初,\hidirikasen{鄭武公}娶于\hidirikasen{申}\footnote{注一},曰\hidirikasen{武姜},生\hidirikasen{莊公}及\hidirikasen{共叔段}。\hidirikasen{莊公}寤生,驚\hidirikasen{姜氏},故名曰\hidirikasen{寤生},遂惡之。愛\hidirikasen{共叔段},欲立之,亟請於\hidirikasen{武公},\hidirikasen{公}弗許。

及莊公即位,為之請\hidirinijusen{制}。公曰:「\hidirinijusen{制},巖邑也。\hidirikasen{虢叔}死焉,佗邑唯命\footnote{再注}。」請京,使居之,謂之\hidirikasen{京城大叔}。

祭仲曰:「都城過百雉,國之害也。先王之制:大都,不過參國之一;中,五之一;小,九之一。今京不度,非制也,君將不堪。」公曰:「姜氏欲之,焉辟害?」對曰:「姜氏何厭之有?不如早為之所,無使滋蔓!蔓,難圖也。蔓草猶不可除,況君之寵弟乎?」公曰:「多行不義,必自斃,子姑待之!」

既而大叔命西鄙、北鄙貳於己。公子呂曰:「國不堪貳,君將若之何?欲與大叔,臣請事之;若弗與,則請除之,無生民心。」公曰:「無庸,將自及。」

大叔又收貳以為己邑,至於廩延。子封曰:「可矣!厚將得眾。」公曰:「不義不暱,厚將崩。」

大叔完聚,繕甲兵,具卒乘,將襲鄭,夫人將啟之。公聞其期曰:「可矣。」命子封帥車二百乘以伐京,京叛大叔段。段入于鄢,公伐諸鄢。五月辛丑,大叔出奔共。

書曰:「鄭伯克段于鄢。」段不弟,故不言弟。如二君,故曰克。稱鄭伯,譏失教之。謂之鄭志。不言出奔,難之也。

遂寘姜氏于城潁,而誓之曰:「不及黃泉,無相見也。」既而悔之。潁考叔為潁谷封人,聞之。有獻於公,公賜之食。食舍肉,公問之。對曰:「小人有母,皆嘗小人之食矣。未嘗君之羹,請以遺之。」公曰:「爾有母遺,繄我獨無!」潁考叔曰:「敢問何謂也?」公語之故,且告之悔。對曰:「君何患焉。若闕地及泉,隧而相見,其誰曰不然?」公從之。

公入而賦:「大隧之中,其樂也融融。」姜出而賦:「大隧之外,其樂也泄泄!」遂為母子如初。

君子曰:「潁考叔,純孝也,愛其母,施及莊公。\hidirinamisen{詩}曰:『孝子不匱,永錫爾類』{,}其是之謂乎?」


\section{周鄭交質}

鄭武公、莊公為平王卿士。王貳于虢,鄭伯怨王。王曰:「無之。」故周鄭交質:王子狐為質於鄭,鄭公子忽為質於周。

王崩,周人將畀虢公政。四月,鄭祭足帥師取溫之麥。秋,又取成周之禾。周鄭交惡。

君子曰:「信不由中,質無益也。明恕而行,要之以禮,雖無有質,誰能間之?苟有明信:澗溪沼沚之毛,蘋蘩薀藻之菜,筐筥錡釜之器,潢汙行潦之水,可薦於鬼神,可羞於王公。而況君子結二國之信,行之以禮,又焉用質?\hidirinamisen{風}有\hidirinamisen{采}\hidirinamisen{蘩}、\hidirinamisen{采蘋},\hidirinamisen{雅}有\hidirinamisen{行葦}、\hidirinamisen{泂酌},昭忠信也。」


\section{石碏諫寵州吁}

衞莊公娶于齊東宮得臣之妹,曰莊姜,美而無子。衞人所為賦\hidirinamisen{碩人}也。又娶于陳,曰厲媯,生孝伯,蚤死。其娣戴媯,生桓公,莊姜以為己子。

公子州吁,嬖人之子也,有寵而好兵,公弗禁。莊姜惡之。

石碏諫曰:「臣聞愛子,教之以義方,弗納於邪。驕、奢、淫、佚,所自邪也。四者之來,寵祿過也。將立州吁,乃定之矣;若猶未也,階之為禍。夫寵而不驕,驕而能降,降而不憾,憾而能眕者,鮮矣。且夫賤妨貴,少陵長,遠間親,新間舊,小加大,淫破義,所謂六逆也;君義,臣行,父慈,子孝,兄愛,弟敬,所謂六順也。去順效逆,所以速禍也。君人者,將禍是務去;而速之,無乃不可乎?」弗聽。

其子厚與州吁遊,禁之,不可。桓公立,乃老。


\chapter{}

\section{鄭子家告趙宣子}

晉侯合諸侯于扈,平宋也。於是晉侯不見鄭伯,以為貳于楚也。 鄭子家使執訊而與之書,以告趙宣子,曰:「寡君即位三年,召蔡侯而與之事君。九月,蔡侯入于敝邑以行;敝邑以侯宣多之難,寡君是以不得與蔡侯偕。十一月,克減侯宣多,而隨蔡侯以朝於執事。

十二年六月,歸生佐寡君之嫡夷,以請陳侯於楚,而朝諸君。十四年七月,寡君又朝以蕆陳事。十五年五月,陳侯自敝邑往朝於君。往年正月,燭之武往朝夷也。八月,寡君又往朝。以陳、蔡之密邇於楚,而不敢貳焉,則敝邑之故也。雖敝邑之事君,何以不免?在位之中,一朝于襄,而再見於君。夷與孤之二三臣,相及於絳。雖我小國,則蔑以過之矣。

今大國曰:『爾未逞吾志。』敝邑有亡,無以加焉!古人有言曰:『畏首畏尾,身其餘幾?』又曰:『鹿死不擇音。』小國之事大國也,德,則其人也;不德,則其鹿也。鋌而走險,急何能擇?命之罔極,亦知亡矣。將悉敝賦,以待於鯈,唯執事命之。

文公二年,朝於齊;四年,為齊侵蔡,亦獲成於楚。居大國之閒,而從於強令,豈其罪也?大國若弗圖,無所逃命。」

晉鞏朔行成於鄭,趙穿、公婿池為質焉。


\end{document}
